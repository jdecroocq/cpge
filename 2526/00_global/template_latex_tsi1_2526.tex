
% Ce template est conçu pour générer des fiches de synthèse uniformes.

% ===================================================================
% PRÉAMBULE
% ===================================================================

\documentclass[a4paper,12pt]{article}

\usepackage[utf8]{inputenc}
\usepackage[T1]{fontenc}
\usepackage[french]{babel}

\usepackage{geometry}
\geometry{
  margin=2cm,
  top=1.2cm,
  bottom=1.8cm,
  includefoot
}
\setlength{\footskip}{2cm}
\usepackage{parskip}

\usepackage{fancyhdr}

\usepackage{amsmath, amssymb}

\usepackage{graphicx}
\usepackage{array}
\usepackage{longtable}
\usepackage{tabularx}
\usepackage{multirow}
\usepackage{xcolor}
\usepackage{listings}
\usepackage[version=4]{mhchem}
\usepackage{bm}

\usepackage{ragged2e}
\usepackage{titlesec}
\titleformat{\section}
  {\large\bfseries}
  {}
  {0em}
  {\underline}

\usepackage{enumitem}
\setlist[itemize]{
    label=\textbf{--},
    topsep=0pt,
    itemsep=0\baselineskip,
    partopsep=0pt,
    leftmargin=*
}
\newlist{subitemize}{itemize}{1}
\setlist[subitemize]{
    label=$\circ$,
    nosep, 
    topsep=-\parskip, 
    leftmargin=*,
    after=\vspace{0pt} 
}

\usepackage{datetime}


% ===================================================================
% COMMANDES PERSONNALISÉES
% ===================================================================

\newcommand{\repere}{%
  00\_global\_%           selon la matière : 00_global, 01_maths, 02__pc, 03_ssi, 04_info, 05_fr, 06_lv
  \twodigit{\month}%
  \twodigit{\day}%
  \twodigit{\currenthour}%
  \twodigit{\currentminute}%
}

% Paramétrage du pied de page.
\pagestyle{fancy}
\fancyhf{}
\fancyfoot[L]{\small 2025--2026}
\fancyfoot[C]{\small \thepage}
\fancyfoot[R]{\small TSI1 -- Jean Decroocq}
\renewcommand{\headrulewidth}{0pt}
\renewcommand{\footrulewidth}{0.4pt}

\newcolumntype{P}[1]{>{\centering\arraybackslash}p{#1}}



\lstset{
    language=Python,
    basicstyle=\ttfamily\footnotesize\color[HTML]{ebcf82},
    numbers=left,
    numberstyle=\color{gray},
    breaklines=true,
    showstringspaces=false,
    frame=none, 
    tabsize=4,
    backgroundcolor=\color[HTML]{252525},
    identifierstyle=\color[HTML]{ffffff},
    keywordstyle=\color[HTML]{4bb5ea},
    stringstyle=\color[HTML]{ed6a68},
    commentstyle=\color[HTML]{75dea0},
    literate={é}{{\'e}}1
           {è}{{\`e}}1
           {à}{{\`a}}1
           {ç}{{\c c}}1
           {ù}{{\`u}}1
           {â}{{\^a}}1
           {ê}{{\^e}}1
           {î}{{\^i}}1
           {ô}{{\^o}}1
           {û}{{\^u}}1
}

% ===================================================================
% DÉBUT DU DOCUMENT
% ===================================================================
\begin{document}

% --- EN-TÊTE DE LA FICHE ---
\noindent
\begin{minipage}[t]{0.15\textwidth}
    \small \repere
\end{minipage}%
\begin{minipage}[t]{0.7\textwidth}
    \centering
    \small Thème principal
\end{minipage}

\vspace{-0.2cm}
\noindent
\begin{center}
{\large \textbf{Titre du document}}
\\[0.3cm]
{\small \textit{Sous-titre optionnel}}
\end{center}
\vspace{0.3cm}

% ===================================================================
% CORPS DE LA FICHE
% ===================================================================


\hspace*{2em}Ceci est un paragraphe de texte normal qui n'est pas dans une liste.
Il est utile pour les introductions ou les explications générales, mais honnêtement peu utilisé dans les fiches de synthèse.


\section{Titre de section}

\begin{itemize}
    \item Ceci est un item de premier niveau.
    Un même item peut contenir plusieurs lignes de texte sans problème.
    Pour revenir à la ligne, on met deux antislash. \\
    C'est ainsi qu'on revient à la ligne dans un même item, pour compléter le propos par exemple.

    \item Ceci est un autre item. Il contient une sous-liste pour détailler des points :
    \begin{subitemize}
        \item Ceci est un sous-item.
        \item Un autre sous-item pour la démonstration.
    \end{subitemize}
    
    \item Dans ce dernier item de la liste, je me permets de faire remarquer qu'on utilise jamais de mise en relief (gras, italique, souligné).
\end{itemize}



\section{Autre section avec des formules}

\begin{itemize}
    \item Une formule mathématique peut être insérée et centrée sur sa propre ligne de cette manière :
     \[
    s(t) = \langle s(t) \rangle + \sum_{n=1}^{\infty} A_n \cos(n\omega t + \varphi_n).
    \]
    On veille toujours à la ponctuation, hautement considérée même dans les formules !
    
    \item Il est possible de présenter plusieurs formules courtes sur une seule ligne pour économiser de l'espace :
    
    \noindent\begin{minipage}{0.33\linewidth}
        \centering \( \displaystyle f = \frac{1}{T} \)
    \end{minipage}%
    \begin{minipage}{0.33\linewidth}
        \centering \( \displaystyle T = \frac{1}{f} \)
    \end{minipage}%
    \begin{minipage}{0.33\linewidth}
        \centering \( \displaystyle \omega = 2\pi f \)
    \end{minipage}
\end{itemize}


\section{Exemples de tableaux et images}

\begin{itemize}
    \item Il est bon d'utiliser des tableaux dès que possible, dont la mise en forme contribue à l'intellect. 

    \item Voici un petit tableau centré, avec des en-têtes non gras.
    \begin{center}
    \begin{tabular}{|c|c|c|c|}
        \hline
        Atome & H & C & N \\
        \hline
        $\chi$ & 2,20 & 2,55 & 3,04 \\
        \hline
    \end{tabular}
    \end{center}

    \item Voici un tableau commun.
    \begin{center}
    \begin{tabular}{|>{\raggedright\arraybackslash}p{4cm}|>{\raggedright\arraybackslash}p{3cm}|>{\raggedright\arraybackslash}p{5cm}|}
    \hline
    \textbf{Catégorie} & \textbf{Formule} & \textbf{Nom} \\
    \hline
    Molécule neutre & \ce{NH3} & Ammoniac \\
    \hline
    \multirow{2}{*}{Cation (positif)} & \ce{H3O+} & Oxonium \\
                                    & \ce{NH4+} & Ammonium \\
    \hline
    \multirow{4}{*}{Anion (négatif)} & \ce{OH-} & Hydroxyde \\
                                    & \ce{SO4^{2-}} & Sulfate \\
                                    & \ce{NO3-} & Nitrate \\
                                    & \ce{MnO4-} & Permanganate \\
    \hline
    \end{tabular}
    \end{center}
    \leavevmode %très important

    \item Et voici un tableau plus conséquent qui prend la largeur d'un paragraphe:

    \renewcommand{\arraystretch}{1.3}
    {\small
    \begin{tabularx}{\linewidth}{|
        >{\justifying\arraybackslash}p{0.2\linewidth}|
        >{\justifying\arraybackslash}X|
        >{\justifying\arraybackslash}X|
    }
        \hline
        \textbf{Source} &
        \textbf{Une colonne} &
        \textbf{Une autre} \\
        \hline
        Exemple &
        Contenu de la case &
        Autre contenu \\
        \hline
        C'est un tableau &
        En effet &
        Un tableau de qualité ! \\
        \hline
        Pour quoi faire ? &
        Idéal lorsqu'il y a pas mal de contenu &
        Pour ne pas s'y perdre... \\
        \hline
    \end{tabularx}
    }
    \renewcommand{\arraystretch}{1}  

    Lorsque le tableau contient du contenu conséquent, il est preferable d'y adopter les normes de ponctuation.

    \item Il est aisé d'insérer une image et de la centrer :
    \begin{center}
        \includegraphics[width=0.5\linewidth]{example-image-a}
    \end{center}    

\end{itemize}


\section{Mise en page avancée}

\noindent
\begin{minipage}[c]{0.55\textwidth}
    \begin{itemize}[itemsep=0.5\baselineskip]
        \item L'environnement \texttt{minipage} permet de créer des blocs de contenu.
        \item On peut ainsi placer une liste d'items à côté d'une illustration, ce qui à la fois fait gagner de la place, et permet d'illustrer directement le propos.
        \item La largeur de chaque bloc est définie en pourcentage. Veillons à ce que la somme ne dépasse pas 1 !
    \end{itemize}
\end{minipage}%
\begin{minipage}[c]{0.45\textwidth}
    \centering
    \includegraphics[width=0.9\linewidth]{example-image-b} 
\end{minipage}



\section{Exemple de code informatique (Python)}

\begin{itemize}
    \item On peut faire référence à des éléments de code directement dans le texte, par exemple en parlant de la fonction \texttt{calculerValeur()} ou de la variable \texttt{resultat}.
    
    \item Pour présenter un algorithme, on utilise un bloc de code. En voici un par exemple :
    
\begin{lstlisting}
def calculer_aire(rayon): # Cette fonction calcule l'aire d'un cercle
    if rayon < 0:
        return "Rayon non valide."
    else:
        aire = 3.14 * (rayon ** 2)
        return aire
\end{lstlisting}
\end{itemize}

\end{document}
