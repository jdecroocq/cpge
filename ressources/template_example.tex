% ===================================================================
% PRÉAMBULE
% ===================================================================
\documentclass[a4paper,12pt]{article}

\usepackage[utf8]{inputenc}
\usepackage[T1]{fontenc}
\usepackage[french]{babel}

\usepackage{geometry}
\geometry{
  margin=2cm,
  top=1.2cm,
  bottom=1.8cm,
  includefoot
}
\setlength{\footskip}{2cm}

\usepackage{fancyhdr}
\usepackage{amsmath, amssymb}
\usepackage{graphicx}
\usepackage{parskip}
\usepackage{array}
\usepackage{titlesec}
\titleformat{\section}
  {\large\bfseries}
  {}
  {0em}
  {\underline}

\usepackage{enumitem}
\setlist[itemize]{
    label=\textbf{--},
    topsep=0pt,
    itemsep=0\baselineskip,
    partopsep=0pt,
    leftmargin=*
}
\newlist{subitemize}{itemize}{1}
\setlist[subitemize]{%
    label=$\circ$,
    nosep, 
    topsep=-\parskip, 
    leftmargin=*,
    after=\vspace{0pt} 
}

\usepackage{datetime}

\newcommand{\repere}{%
  02\_pc\_%
  \twodigit{\month}%
  \twodigit{\day}%
  \twodigit{\currenthour}%
  \twodigit{\currentminute}%
}

\pagestyle{fancy}
\fancyhf{}
\fancyfoot[L]{\small 2025--2026}
\fancyfoot[C]{\small \thepage}
\fancyfoot[R]{\small TSI1 -- Jean Decroocq}
\renewcommand{\headrulewidth}{0pt}
\renewcommand{\footrulewidth}{0.4pt}
\newcolumntype{P}[1]{>{\centering\arraybackslash}p{#1}}

% ===================================================================
% DÉBUT DU DOCUMENT
% ===================================================================
\begin{document}

\noindent
\begin{minipage}[t]{0.15\textwidth}
    \small \repere
\end{minipage}%
\begin{minipage}[t]{0.7\textwidth}
    \centering
    \small 01. Signaux Physiques
\end{minipage}

\vspace{-0.2cm}
\noindent
\centering
{\large \textbf{02. Formation des images}}

\vspace{0.5cm}

% ===================================================================
% CONTENU DE LA FICHE
% ===================================================================

\section{Photons}

\begin{itemize}
    \item Dualité onde-corpuscule : la lumière se comporte à la fois comme une onde (phénomènes de diffraction, interférences) et comme un flux de particules, les photons (effet photoélectrique).

    \item Photon : particule sans masse, se déplaçant à la vitesse de la lumière $c$, et transportant une énergie quantifiée (un quantum d'énergie).
    \[ E_{\text{photon}} = h \nu = h \cdot \frac{c}{\lambda}. \]
    avec $h \approx 6,63 \times 10^{-34}\,\text{J.s}$ (constante de Planck).
    
    \item Électronvolt (eV) : unité d'énergie adaptée à l'échelle atomique.
    \[ 1\,\text{eV} = 1,6 \times 10^{-19}\,\text{J}. \]

    \item Puissance lumineuse : énergie transportée par unité de temps. Elle est liée au flux de photons $\Phi$ (nombre de photons par seconde).
    \[ P = \frac{\Delta E}{\Delta t} = \Phi \cdot E_{\text{photon}}. \]
\end{itemize}


\section{Sources et spectres}

\begin{itemize}
    \item Source primaire : corps qui produit et émet sa propre lumière.
    
    \item Source secondaire : objet qui diffuse la lumière qu'il reçoit.
    
    \item Spectre continu : contient toutes les longueurs d'onde sur une large plage.
    
    \item Spectre de raies (ou discret, ou discontinu) : ne contient que quelques longueurs d'onde discrètes et bien définies.

    \item Lumière monochromatique : ne contient qu'une seule longueur d'onde.

\end{itemize}


\section{Propagation de la lumière}

\begin{itemize}
    \item MHTI : milieu homogène (mêmes propriétés en tout point), transparent (laisse passer la lumière) et isotrope (mêmes propriétés dans toutes les directions) assurant la propagation rectiligne de la lumière.

    \item Indice de réfraction : grandeur sans dimension qui caractérise un milieu.
    \[ n = \frac{c_0}{c} \ge 1, \]
    \newlength{\colwidth}
    \setlength{\colwidth}{1.2cm}
    \begin{center}
    \begin{tabular}{|p{3cm}|P{\colwidth}|P{\colwidth}|P{\colwidth}|P{\colwidth}|P{\colwidth}|}
        \hline
        \textbf{Matière} & Vide & Air & Eau & Verre  \\
        \hline
        \textbf{Indice optique} & $1$ & $1,003$ & $1,33$ & $\sim 1,5$  \\
        \hline
    \end{tabular}
    \end{center}

    \item La fréquence $\nu$ d'une onde lumineuse ne change jamais lors d'un changement de milieu. La longueur d'onde, elle, est modifiée :
    \[ \lambda = \frac{\lambda_0}{n}. \]
\end{itemize}


\section{Lois de Snell-Descartes}

\noindent
\begin{minipage}{\textwidth}
    \begin{minipage}[c]{0.6\textwidth}
        \begin{itemize}[itemsep=0.5\baselineskip]
            \item Réflexion :
                \[ i_1 = i_1' .\]
                
            \item Réfraction :
                \[ n_1 \sin i_1 = n_2 \sin i_2 .\]
                
            \item Réflexion totale : possible uniquement lors du passage d'un \textbf{milieu d'indice plus élevé à un milieu d'indice plus faible}, si l'angle d'incidence est supérieur à l'angle limite $i_{\text{lim}}$.
                \[i_1 > i_{\text{lim}} = \arcsin\left(\frac{n_2}{n_1}\right) .\]
                (Démo. loi de réfraction avec $i_2 = 90^\circ$). \\
                Ex. fibre optique (guidage par réflexions successives), ou vignettage sous l'eau (réciproque, flèches opposées sur le schéma).
                
                
            
            \item Dioptre : surface séparant deux milieux d'indices de réfraction différents.
        \end{itemize}
    \end{minipage}%
    \begin{minipage}[c]{0.4\textwidth}
        \centering
        \includegraphics[width=0.95\linewidth]{descartes.pdf}
        
        \rule{0pt}{0.1cm}
        
        \includegraphics[width=0.95\linewidth]{refl_totale.pdf}
    \end{minipage}
\end{minipage}


\section{Systèmes optiques et lentilles}

\begin{itemize}
    \item Les rayons lumineux qui proviennent d'un point situé à l'infini sont parallèles entre eux.
    
    \item Stigmatisme : un système est rigoureusement stigmatique s'il donne de tout point objet A un unique point image A'. \\
    Le stigmatisme est approché si la taille de la tache image est plus petite que le grain du détecteur (ex. photosite d'un capteur CMOS).

    \item Lentille mince convergente (bords minces, symb. $\updownarrow$) et divergente (bords épais, symb. opposé).
    
    \item Conditions de Gauss (base de toute l'optique géométrique simplifiée): rayons lumineux peu inclinés sur l'axe optique et peu éloignés de celui-ci.
    
    \item Propriétés des lentilles (schémas dans le cours) :
    \renewcommand{\arraystretch}{1.4}
    \begin{center}
    \begin{tabular}{| p{5cm} | p{5.1cm} | p{5.1cm} |}
        \hline
        \textbf{Source} & \textbf{Lentille convergente} & \textbf{Lentille divergente} \\
        \hline
        Rayon passant par le centre optique O. & Rayon non dévié. & Rayon non dévié. \\
        \hline
        Rayon incident parallèle à l'axe optique. & Émerge en passant par le point focal image F'. & Émerge en semblant provenir du point focal image F'. \\
        \hline
        Rayon dont le support passe par le point focal objet F. & Émerge parallèlement à l'axe optique. & Émerge parallèlement à l'axe optique. \\
        \hline
        Faisceau de rayons parallèles entre eux. & Converge en un point du plan focal image. & Diverge en semblant provenir d'un point du plan focal image. \\
        \hline
        Objet placé dans le plan focal objet. & Image à l'infini, rayons émergents parallèles. & Image à l'infini, rayons émergents parallèles. \\
        \hline
    \end{tabular}
    \end{center}
    \renewcommand{\arraystretch}{1}

    \item Un objet est réel s'il est la source physique des rayons ; 
    il est virtuel s'il est situé là où des rayons allaient converger. \\
    Une image est réelle si on peut la former sur un écran (ex. capteur photo après objo, foyer d'un miroir de télescope); 
    elle est virtuelle si elle n'est visible qu'à travers le système optique, par prolongement des rayons (ex. loupe, miroir).
    
    \item L'image d'un objet perpendiculaire à l'axe optique est elle-même perpendiculaire à cet axe.

    \item Distance focale $f'$, vergence $V$ :
    \[ f' = \overline{OF'} = - \overline{OF}, \qquad V = \frac{1}{f'}. \]
    $f' > 0$ pour une lentille convergente, $f' < 0$ pour une divergente. $V$ s'exprime en dioptries ($\delta$).
    
    \item Relation de conjugaison :
    \[ \frac{1}{\overline{OA'}} = \frac{1}{\overline{OA}} + \frac{1}{\overline{OF'}}; \qquad \qquad \overline{OA'} = \frac{\overline{OA} \times \overline{OF'}}{\overline{OA} + \overline{OF'}}. \]
    $\overline{OA}$ est très souvent négatif (objet réel placé avant la lentille).
    
    \item Grandissement transversal (différent du grossissement) :
    \[ \gamma = \frac{\overline{A'B'}}{\overline{AB}} = \frac{\overline{OA'}}{\overline{OA}}. \]
    Si $\gamma > 0$, l'image est droite. Si $\gamma < 0$, l'image est renversée. \\
    Si $|\gamma| > 1$, l'image est agrandie. Si $|\gamma| < 1$, l'image est rétrécie.
    
    \item Focométrie : méthodes expérimentales pour déterminer la distance focale d'une lentille (voir cours).
    
\end{itemize}

\end{document}
